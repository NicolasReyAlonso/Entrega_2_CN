\section{Demostraciones de Funcionalidad en AWS}

A continuación se presentan capturas de pantalla que demuestran el correcto funcionamiento de la arquitectura implementada.

Cabe destacar que, debido a que la practica la he realizado en su totalidad en la 
terminal con ".sh" y ".py", las imágenes corresponden a capturas de los resultados,
no al proceso de creación de la arquitectura. Para la creación de la misma, 
vaya a la sección \textit{Desarrollo de las actividades}.

\subsection{Amazon S3 - Data Lake}

\begin{figure}[H]
\centering
\includegraphics[width=0.9\textwidth]{Imagenes/0 Bucket.png}
\caption{Bucket S3 creado para el Data Lake}
\label{fig:bucket}
\end{figure}

\begin{figure}[H]
\centering
\includegraphics[width=0.9\textwidth]{Imagenes/1 datalake.png}
\caption{Estructura de carpetas del Data Lake en S3}
\label{fig:datalake}
\end{figure}

\begin{figure}[H]
\centering
\includegraphics[width=0.9\textwidth]{Imagenes/2 Datalake processed 1.png}
\caption{Datos procesados en la capa processed - Vista 1}
\label{fig:processed1}
\end{figure}

\begin{figure}[H]
\centering
\includegraphics[width=0.9\textwidth]{Imagenes/2 Datalake processed 2.png}
\caption{Datos procesados en la capa processed - Vista 2}
\label{fig:processed2}
\end{figure}

\subsection{AWS Glue - Catálogo de Datos}

\begin{figure}[H]
\centering
\includegraphics[width=0.9\textwidth]{Imagenes/3 databases.png}
\caption{Base de datos creada en AWS Glue Data Catalog}
\label{fig:databases}
\end{figure}

\begin{figure}[H]
\centering
\includegraphics[width=0.9\textwidth]{Imagenes/4 Tables.png}
\caption{Tablas detectadas por el Crawler en el catálogo}
\label{fig:tables}
\end{figure}

\begin{figure}[H]
\centering
\includegraphics[width=0.9\textwidth]{Imagenes/5 schema.png}
\caption{Esquema inferido automáticamente por el Crawler}
\label{fig:schema}
\end{figure}

\begin{figure}[H]
\centering
\includegraphics[width=0.9\textwidth]{Imagenes/6 Partitions.png}
\caption{Particiones detectadas por año de lanzamiento}
\label{fig:partitions}
\end{figure}

\begin{figure}[H]
\centering
\includegraphics[width=0.9\textwidth]{Imagenes/7 glue crawlers.png}
\caption{Crawler configurado y ejecutado correctamente}
\label{fig:crawlers}
\end{figure}

\subsection{Amazon Athena - Consultas}

\begin{figure}[H]
\centering
\includegraphics[width=0.9\textwidth]{Imagenes/8 queries 1.png}
\caption{Consulta SQL en Athena sobre los datos del Data Lake - Ejemplo 1}
\label{fig:queries1}
\end{figure}

\begin{figure}[H]
\centering
\includegraphics[width=0.9\textwidth]{Imagenes/8 queries 2.png}
\caption{Consulta SQL en Athena sobre los datos del Data Lake - Ejemplo 2}
\label{fig:queries2}
\end{figure}

\subsection{AWS Glue - Jobs ETL}

\begin{figure}[H]
\centering
\includegraphics[width=0.9\textwidth]{Imagenes/9 ETL.png}
\caption{Jobs ETL ejecutados correctamente en AWS Glue}
\label{fig:etl}
\end{figure}