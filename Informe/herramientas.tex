\section{Herramientas Utilizadas}
\subsection{Desarrollo}
Para el desarrollo de la práctica se han utilizado las siguientes herramientas:
\begin{itemize}
    \item \textbf{AWS CLI:} Herramienta de línea de comandos para interactuar con 
    los servicios de AWS, utilizada para automatizar la creación y configuración 
    de recursos.
    \item \textbf{Python 3.12:} Lenguaje de programación empleado para desarrollar 
    los scripts del productor de datos, funciones Lambda y jobs ETL en Glue.
    \item \textbf{Boto3:} SDK de AWS para Python, utilizado para interactuar 
    programáticamente con los servicios de AWS desde los scripts.
    \item \textbf{Visual Studio Code:} Entorno de desarrollo integrado (IDE) 
    utilizado para escribir y depurar el código Python, además de redactar la 
    documentación.
    \item \textbf{GitHub Copilot:} Asistente de programación basado en inteligencia 
    artificial que ha ayudado en la generación de código durante el desarrollo.
\end{itemize}
\subsection{Documentación}
Para la elaboración del informe y la documentación de la práctica se han 
utilizado las siguientes herramientas:
\begin{itemize}
    \item \textbf{LaTeX:} Sistema de preparación de documentos utilizado para 
    redactar el informe técnico de la práctica.
    \item \textbf{GitHub Copilot:} Asistente de inteligencia artificial que ha 
    colaborado en la redacción y estructuración del informe.
\end{itemize}
\subsection{Gestión de versiones}
Para el control de versiones y la gestión del código fuente se ha utilizado:
\begin{itemize}
    \item \textbf{Git:} Sistema de control de versiones distribuido utilizado para 
    gestionar los cambios en el código y la documentación.
    \item \textbf{GitHub:} Plataforma de alojamiento de repositorios Git, utilizada para 
    almacenar el código fuente y colaborar en el desarrollo.
\end{itemize}
\subsection{}