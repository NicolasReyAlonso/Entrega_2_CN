% Compilar con XeLaTeX: xelatex P2_Nicolás_Rey_Alonso.tex
% Renombrar salida a: P2_Apellido1_Apellido2_Nombre.pdf
% Reemplaza los valores de \Asignatura y \Practica abajo.
% Requisitos implementados:
% - Times New Roman 12pt (cuerpo)
% - Arial negrita para títulos (14/13/12 pt)
% - Interlineado 1.5
% - Márgenes: superior/inferior/derecho 2.5 cm, izquierdo 3 cm
% - Texto justificado
% - Encabezado (derecha): nombre de asignatura y práctica
% - Pie centrado: Página X de Y

% Usar XeLaTeX o LuaLaTeX para garantizar las fuentes exactas.
\documentclass[a4paper,12pt]{article}
\usepackage{fontspec}
\setmainfont{Times New Roman}
\newfontfamily\headingfont{Arial}

\usepackage[a4paper,left=3cm,right=2.5cm,top=2.5cm,bottom=2.5cm]{geometry}
\usepackage{setspace}
\onehalfspacing

\usepackage{titlesec}
% H1: 14pt bold Arial
\titleformat{\section}{\headingfont\bfseries\fontsize{14}{16}\selectfont}{\thesection}{1em}{}
% H2: 13pt bold Arial
\titleformat{\subsection}{\headingfont\bfseries\fontsize{13}{15}\selectfont}{\thesubsection}{1em}{}
% H3: 12pt bold Arial
\titleformat{\subsubsection}{\headingfont\bfseries\fontsize{12}{14}\selectfont}{\thesubsubsection}{1em}{}

\usepackage{microtype}
\usepackage{lastpage}
\usepackage{fancyhdr}
\setlength{\headheight}{15pt}
\pagestyle{fancy}
\fancyhf{}
% Encabezado a la derecha: asignatura y práctica
\newcommand{\Asignatura}{Computación en La Nube}
\newcommand{\Practica}{Práctica 2}
\fancyhead[R]{\Asignatura\ --\ \Practica}
% Pie centrado: Página X de Y
\fancyfoot[C]{Página \thepage\ de \pageref{LastPage}}
\renewcommand{\headrulewidth}{0pt}
\renewcommand{\footrulewidth}{0pt}

% Asegurar texto justificado (por defecto en LaTeX)
\sloppy

% Metadatos opcionales
\usepackage{hyperref}
\hypersetup{pdfauthor={Autor},pdftitle={Informe}}

% Documento
\begin{document}

% Portada mínima — edita según necesidad
\begin{titlepage}
  \centering
  {\huge\bfseries Informe\par}
  \vspace{2cm}
  {\Large \Asignatura\par}
  {\large \Practica\par}
  \vfill
  {\large Autor: Nombre Apellido\par}
  {\large Fecha: \today\par}
\end{titlepage}

\tableofcontents
\newpage

\section{Introducción}
El objetivo de esta práctica es construir una arquitectura en la nube que a partir
de un flujo de datos con kinesis realice un procesamiento en tiempo real y 
almacene los resultados en una base de datos generada automaticamente a trvés de 
un glue crawler.

\subsection{Subsección de ejemplo}
Contenido de la subsección.

\subsubsection{Subsubsección}
Contenido de la subsubsección.

\section{Resultados}
Incluye tablas, figuras y explicaciones aquí.

\section{Conclusiones}
Conclusiones del trabajo.

\end{document}
